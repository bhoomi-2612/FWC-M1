\documentclass[a4paper,12pt]{article}

\usepackage{amsmath,amssymb}
\usepackage{geometry}
\usepackage{enumitem}
\usepackage{fancyhdr}
\usepackage{graphicx}

\geometry{top=1in,bottom=1in,left=1in,right=1in}
\pagestyle{plain}

\begin{document}

\thispagestyle{fancy}
\fancyhf{}
\fancyhead[L]{
    \includegraphics[width=8cm,height=1.7cm]{logo.png}
}
\fancyhead[R]{
    Name: Bhoomika B V \\
    Id: COMETFWC046 \\
    Date: 2nd February 2026
}
\renewcommand{\headrulewidth}{0pt}
\fancyfoot[C]{\thepage}

\vspace*{0.2cm}

\begin{center}
\textbf{\Large 27$^{\text{th}}$ International Mathematical Olympiad}\\
Warsaw, Poland\\
Day I\\
July 9, 1986
\end{center}

\vspace{1cm}

\begin{enumerate}

\item Let $d$ be any positive integer not equal to 2, 5, or 13. 
Show that one can find distinct $a,b$ in the set $\{2,5,13,d\}$ such that $ab-1$ is not a perfect square.

\item A triangle $A_1A_2A_3$ and a point $P_0$ are given in the plane.  
Define $A_n = A_{n-3}$ for $n \geq 4$.  
Construct $P_n$ from $P_{n-1}$ by rotation through angle $120^\circ$ clockwise about $A_n$.  
Prove that if $P_{1986} = P_0$, then triangle $A_1A_2A_3$ is equilateral.

\item To each vertex of a regular pentagon an integer is assigned such that the sum of all five numbers is positive.  
If three consecutive vertices are assigned $x,y,z$ respectively and $y<0$, then replace them by $x+y$, $-y$, $z+y$.  
Determine whether this procedure necessarily ends after finitely many steps.

\end{enumerate}

\begin{center}
Day II\\
July 10, 1986
\end{center}

\vspace{1cm}

\begin{enumerate}
\setcounter{enumi}{3}

\item Let $A,B$ be adjacent vertices of a regular $n$-gon ($n\ge5$).  
A triangle $XYZ$ congruent to $OAB$ moves such that $Y,Z$ trace the boundary of the polygon while $X$ remains inside.  
Find the locus of $X$.

\item Find all functions $f$ on non-negative real numbers satisfying:
\begin{enumerate}[label=(\roman*)]
\item $f(xf(y)) = f(x+y)$
\item $f(2)=0$
\item $f(x)\ne0$ for $0\le x<2$
\end{enumerate}

\item Given a finite set of lattice points in the plane, is it always possible to color them red and white so that for every line parallel to axes, the difference in counts on the line is at most 1?

\end{enumerate}


\begin{center}
\textbf{\Large 28$^{\text{th}}$ International Mathematical Olympiad}\\
Havana, Cuba\\
Day I\\
July 10, 1987
\end{center}

\vspace{1cm}

\begin{enumerate}

\item Let $p_n(k)$ be number of permutations of $\{1,\dots,n\}$ with exactly $k$ fixed points.  
Prove:
\[
\sum_{k=0}^{n} k\, p_n(k) = n!
\]

\item In an acute triangle $ABC$, let internal bisector of $\angle A$ meet $BC$ at $L$ and circumcircle again at $N$.  
From $L$, drop perpendiculars to $AB, AC$.  
Prove quadrilateral $AKNM$ and triangle $ABC$ have equal areas.

\item Let $x_1,\dots,x_n$ satisfy $x_1^2+\cdots+x_n^2=1$.  
Prove there exist integers $a_1,\dots,a_n$, not all zero, with
\[
|a_i|\le k-1
\]
and
\[
|a_1x_1+\cdots+a_nx_n|\le \frac{(k-1)\sqrt n}{k^n-1}.
\]

\end{enumerate}


\begin{center}
Day II\\
July 11, 1987
\end{center}

\vspace{1cm}

\begin{enumerate}
\setcounter{enumi}{3}

\item Prove there is no function $f:\mathbb{N}\to\mathbb{N}$ such that $f(f(n)) = n + 1987$.

\item Let $n\ge3$. Prove there exists $n$ points in plane such that all pairwise distances are irrational and every triangle formed has rational area.

\item Let $n\ge2$. Prove $n^2+k+n$ is prime for all integers $k$ with $0\le k\le \sqrt{n/3}$ implies $k^2+k+n$ is prime for $0\le k\le n-2$.

\end{enumerate}


\begin{center}
\textbf{\Large 29$^{\text{th}}$ International Mathematical Olympiad}\\
Canberra, Australia\\
Day I
\end{center}

\vspace{1cm}

\begin{enumerate}

\item Consider two concentric circles of radii $R>r$.  
Let $P$ be on smaller circle and $B$ on larger.  
If perpendicular to $BP$ at $P$ meets smaller circle again at $A$,  
find:
\begin{enumerate}[label=(\roman*)]
\item $BO^2 + CA^2 + AB^2$
\item Locus of midpoint of $BC$
\end{enumerate}

\item Let $A_1,\dots,A_{2n+1}$ be subsets of a set $B$ satisfying certain intersection properties.  
For which $n$ can each element of $B$ be assigned 0 or 1 so that each $A_i$ has exactly $n$ zeros?

\item Function $f$ satisfies:
\[
f(1)=1,\quad f(3)=3
\]
\[
f(2n)=f(n)
\]
\[
f(4n+1)=2f(2n+1)-f(n)
\]
\[
f(4n+3)=3f(2n+1)-2f(n)
\]
Determine number of $n\le1988$ with $f(n)=n$.

\end{enumerate}


\begin{center}
Day II
\end{center}

\vspace{1cm}

\begin{enumerate}
\setcounter{enumi}{3}

\item Show solution set of
\[
\sum_{k=1}^{70} \frac{k}{x-k} \ge \frac{5}{4}
\]
is union of disjoint intervals of total length 1988.

\item In right triangle $ABC$, prove $S\ge2T$ where $S,T$ are areas of certain triangles formed by incenters.

\item Let $a,b$ be positive integers such that $ab+1$ divides $a^2+b^2$.  
Show
\[
\frac{a^2+b^2}{ab+1}
\]
is a perfect square.

\end{enumerate}

\end{document}

